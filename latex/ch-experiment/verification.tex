\section{Verification Performance}
\label{sec:experiment:verification}

The database was divided into a training part (the first session of 4000 samples) and a testing part (the second session of 4000 samples). As described in Section 3, the dimension of the proposed global features is  . To select the value of M and N we carried out a series of verifications on the training database where the class of the input palmprint was known. Each of the 3D samples was matched with the remaining samples in the training database. A successful match is where the two samples are from the same class. This is referred to as intra-class matching and the candidate image is said to be genuine. An unsuccessful match is referred to as inter-class matching and the candidate image is said to be an impostor. Treating the global features as a point in the   dimension space, we simply use the Euclidian distance as the matching score. Table 1 shows the Equal Error Rate (EER) for   and  . The best result is   and  .

In order to balance accuracy and efficiency, we chose   and   in the following experiments. This means the global features have   dimensions. Table 2 shows the verification results by MD, HCA, RLL and their combined results. From the last column of Table 2 we can see using the combined three global features will achieve a lower EER than each of the individual features. As described in section 3.1, we use the OLDA method to reduce global features to a lower   dimension. To decide the optimal value of  , we carried out a series of recognition experiments on the 4000 sample training database. We divided this database into two equal parts and then chose the first five samples of every palm for training and set aside the rest for testing. As shown in section 3.1,   is equal to   in (13). Instead of  , we let  .