\section{Identificaiton Performance}
\label{sec:experiment:identificaiton}

Table 3 and Fig. 10 show the recognition results. We can see that 15 dimensions is a good choice for the following coarse-level matching and RSVM schemes.

Fig. 11 shows the genuine and imposter distributions when the 3D palmprint 15-dimensional global features are applied to the 4000 samples in the training database. Figs. 11 (a) to (o) are obtained by using the Euclidian distance to match the single dimension value from the 1st to 15th. Fig. 11 (p) shows the result of using the Euclidian distance to match all 15 dimensional values.
We next carried out the 3D palmprint classification and recognition experiments using the first sample of each class in the training database as a template and the 4000 samples in the testing database as probes, making a total of 400 templates and 4000 probes. The performance of classification and recognition is usually measured by error rate and penetration rate calculated in [19] as follows:
                               (24)
           (25)


         (a)                   (b)                    (c)                   (d)

         (e)                    (f)                   (g)                   (h)

          (i)                   (j)                    (k)                  (l)

          (m)                   (n)                   (o)                   (p)
Figure 11. Genuine and imposter distributions by the 3D palmprint 15-dimensional global features. (a) to (o) are obtained by matching the single dimension value from the 1st to 15th and (p) is obtained by matching all 15 dimensional values together.

Obviously there is a trade-off between error rates and penetration rates. Generally speaking, if there is no classification, there are two retrieval strategies: 1) all of the templates in the database are visited and the template that gives the best matching score is regarded as the matched template, if the matching score is less than a given threshold  ; 2) given a threshold  , the search continues until a match is found that is below that threshold.
We used three 3D palmprint recognition matching approaches: 1) no classification; 2) coarse-level matching; and 3) RSVM. For no classification, we matched using the local feature MCI as described in [13]. The process we used for coarse-level matching is illustrated in Fig. 9 and involves fine-level matching using the local feature MCI. A single instance of coarse-level matching requires only 1/36000 of the time it takes to do fine-level matching (coarse-level matching only needs 15 operations while fine-level matching must do   operations, where   is the size of ROI and   is the shifting template times). For the above two approaches, the penetration rate and the error rate will vary with different thresholds  . As for RSVM, we use the RSVM algorithm described in section 3.3 to rank the templates in the database, and then match the top   percent by local feature MCI with the best matching score regarded as the matched template if this score is less than a given constant threshold  . We can see from (25) that the   is equal to the penetration rate. Given different thresholds   and  , we carried out a series of 3D palmprint recognition experiments. Table 4, Table 5 and Fig. 12 show these experimental results. Even at an approximately equal error rate, the proposed coarse-level matching and RSVM approaches get a much lower penetration rate than the no classification approach. Obviously RSVM has the best performance but requires an additional offline training process compared to coarse-level matching.
