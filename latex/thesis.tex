% ****************************************************************************************** % Dissertation template and document class for The Hong Kong Polytechnic University
% Author  : QING Pei <edwardtoday@gmail.com>
% Adapted from: http://www.math.princeton.edu/graduate/tex/puthesis.html
% ****************************************************************************************** %


%%% For print copies
%% set 'singlespace' option to set entire thesis to single space, and define "\printmode" to remove all hyperlinks for printed copies of the thesis. Delete all output files before changing this mode -- it will turn hyperref package on and off
%\documentclass[12pt,lot, lof, singlespace]{hkputhesis}
%\newcommand{\printmode}{}

%%% For the electronic copy, use doublespacing, define "\proquestmode" to use outlined links, instead of colored links. 
\documentclass[12pt,lot,lof]{hkputhesis}
\newcommand{\proquestmode}{}
% I prefer proquestmode to be off for electronic copies for normal use, since the colored links are less distracting. However when printed in black and white, the colored links are difficult to read. 

%%% For early drafts without some of the frontmatter
% Also see the "ifodd" command below to disable more frontmatter
%\documentclass[12pt]{hkputhesis}

%%%%%%%%%%%%%%%%%%%%%%%%%%%%%%%%%%%%%%%%%%%%%%%%%%%%%%%%%%%%%\
%%%% Author & title page info

\title{Palmprint Recognition with Three Dimensional Features}

\submitted{June 2012}  % degree conferral date (January, April, June, September, or November)
\copyrightyear{2012}  % year in which the copyright is secured by publication of the dissertation.
\author{QING Pei}
\adviser{Professor David Zhang}  %replace with the full name of your adviser
%\departmentprefix{Program in}  % defaults to "Department of", but programs need to change this.
\department{Computing}

%%%%%%%%%%%%%%%%%%%%%%%%%%%%%%%%%%%%%%%%%%%%%%%%%%%%%%%%%%%%%\
%%%% Tweak float placements
% From: http://mintaka.sdsu.edu/GF/bibliog/latex/floats.html "Controlling LaTeX Floats"
% and based on: http://www.tex.ac.uk/cgi-bin/texfaq2html?label=floats
% LaTeX defaults listed at: http://people.cs.uu.nl/piet/floats/node1.html

% Alter some LaTeX defaults for better treatment of figures:
    % See p.105 of "TeX Unbound" for suggested values.
    % See pp. 199-200 of Lamport's "LaTeX" book for details.
    %   General parameters, for ALL pages:
    \renewcommand{\topfraction}{0.85}	% max fraction of floats at top
    \renewcommand{\bottomfraction}{0.6}	% max fraction of floats at bottom
    %   Parameters for TEXT pages (not float pages):
    \setcounter{topnumber}{2}
    \setcounter{bottomnumber}{2}
    \setcounter{totalnumber}{2}     % 2 may work better
    \setcounter{dbltopnumber}{2}    % for 2-column pages
    \renewcommand{\dbltopfraction}{0.66}	% fit big float above 2-col. text
    \renewcommand{\textfraction}{0.15}	% allow minimal text w. figs
    %   Parameters for FLOAT pages (not text pages):
    \renewcommand{\floatpagefraction}{0.66}	% require fuller float pages
	% N.B.: floatpagefraction MUST be less than topfraction !!
    \renewcommand{\dblfloatpagefraction}{0.66}	% require fuller float pages

% The documentclass already sets parameters to make a high penalty for widows and orphans. 

%%%%%%%%%%%%%%%%%%%%%%%%%%%%%%%%%%%%%%%%%%%%%%%%%%%%%%%%%%%%%\
%%%% Use packages

\usepackage{amsfonts,amsmath}

%%% For figures
\usepackage{graphicx}
\usepackage{epstopdf}

\usepackage{subfigure}
%\usepackage{subfig,rotate}

%%% for comments
\usepackage{verbatim}

%%% for code
\usepackage{minted}

%%% For tables
\usepackage{multirow}
% Longtable lets you have tables that span multiple pages.
\usepackage{longtable}
\usepackage{lscape}
\usepackage[tableposition=top]{caption}
\captionsetup[table]{labelsep=period}

% Booktabs produces far nicer tables than the standard LaTeX tables.
%   see: http://en.wikibooks.org/wiki/LaTeX/Tables
\usepackage{booktabs}

%set parameters for longtable:
% default caption width is 4in for longtable, but wider for normal tables
\setlength{\LTcapwidth}{\textwidth}

%%% Set fonts
\usepackage{mathptmx} % defines Adobe Times Roman (or equivalent) as default text font
%\usepackage{pslatex} % basically a merger of the times and the (obsolete) mathptm packages

%%% Set line spacing
\usepackage{setspace}
\onehalfspacing

%%% Turn off indentation globally
\usepackage{parskip}
\setlength{\parskip}{17pt} % to simulate a blank line between paragraphs

%%% Page number at bottom right
\usepackage{fancyhdr}
\fancyhf{} % clear all header and footers
\renewcommand{\headrulewidth}{0pt} % remove the header rule
\rfoot{\fancyplain{\thepage\hspace{1.8em}}{\thepage\hspace{1.8em}}}
\pagestyle{fancyplain}

%%% Setting the margins
\usepackage[top=1in, left=1.8in, bottom=1.8in, right=1in]{geometry}

%%% Change chapter heading format
\usepackage{shadowtext}
\usepackage{titlesec}
\makeatletter
\renewcommand{\@makechapterhead}[1]{%
\vspace*{24pt}{\setlength{\parindent}{0pt} \raggedright \normalfont
\bfseries\chaptersize \shadowtext{\chaptertitlename\ \thechapter \quad\ #1
}}\vspace*{12pt}}
\makeatother
\titleformat{\section}{\normalfont\sectionsize\bfseries}{\thesection}{1em}{}
\titlespacing*{\section}{0pt}{18pt}{12pt}
\titleformat{\subsection}{\normalfont\titlesize\bfseries}{\thesubsection}{1em}{}
\titlespacing*{\subsection}{0pt}{12pt}{12pt}
\titleformat{\subsubsection}{\normalfont\subsubsectionsize\bfseries}{\thesubsection}{1em}{}
\titlespacing*{\subsubsection}{0pt}{12pt}{12pt}

%%% Changes to table of contents
\renewcommand\contentsname{Table of Contents}
\usepackage[subfigure]{tocloft}
% \setlength{\cftchapnumwidth}{0pt}
% \setlength{\cftbeforechapskip}{\baselineskip}
\renewcommand\cftchappresnum{Chapter } % prefix "Chapter " to chapter number in ToC
\cftsetindents{chapter}{0em}{5em}      % set amount of indenting
\renewcommand{\cftchapdotsep}{\cftdotsep}

\makeatletter
\newcommand\appendix@chapter[1]{%
  \refstepcounter{chapter}%
  \orig@chapter*{Appendix \@Alph\c@chapter: #1}%
  \addcontentsline{toc}{chapter}{Appendix \@Alph\c@chapter: #1}%
}
\let\orig@chapter\chapter
\g@addto@macro\appendix{\let\chapter\appendix@chapter}
\makeatother

%%% Change bibname to References
\renewcommand*{\bibname}{References}

%%%%%%%%%%%%%%%%%%%%%%%%%%%%%%%%%%%%%%%%%%%%%%%%%%%%%%%%%%
%%% Printed vs. online formatting
\ifdefined\printmode

% Printed copy
% url package understands urls (with proper line-breaks) without hyperlinking them
\usepackage{url}


\else

\ifdefined\proquestmode
%ProQuest copy -- http://www.princeton.edu/~mudd/thesis/Submissionguide.pdf

% ProQuest requires a double spaced version (set previously). They will take an electronic copy, so we want links in the pdf, but also copies may be printed or made into microfilm in black and white, so we want outlined links instead of colored links.
\usepackage{hyperref}
\hypersetup{bookmarksnumbered}

% copy the already-set title and author to use in the pdf properties
\makeatletter
\hypersetup{pdftitle=\@title,pdfauthor=\@author}
\makeatother

\else
% Online copy

% adds internal linked references, pdf bookmarks, etc

% turn all references and citations into hyperlinks:
%  -- not for printed copies
% -- automatically includes url package
% options:
%   colorlinks makes links by coloring the text instead of putting a rectangle around the text.
\usepackage{hyperref}
\hypersetup{colorlinks,bookmarksnumbered}

% copy the already-set title and author to use in the pdf properties
\makeatletter
\hypersetup{pdftitle=\@title,pdfauthor=\@author}
\makeatother

% make the page number rather than the text be the link for ToC entries
%\hypersetup{linktocpage}
\fi % proquest or online formatting
\fi % printed or online formatting


%%%%%%%%%%%%%%%%%%%%%%%%%%%%%%%%%%%%%%%%%%%%%%%%%%%%%%%%%%%%%\
%%%% Define commands

% Define any custom commands that you want to use.
% For example, highlight notes for future edits to the thesis
%\newcommand{\todo}[1]{\textbf{\emph{TODO:}#1}}


% create an environment that will indent text
% see: http://latex.computersci.org/Reference/ListEnvironments
% 	\raggedright makes them left aligned instead of justified
\newenvironment{indenttext}{
    \begin{list}{}{ \itemsep 0in \itemindent 0in
    \labelsep 0in \labelwidth 0in
    \listparindent 0in
    \topsep 0in \partopsep 0in \parskip 0in \parsep 0in
    \leftmargin 1em \rightmargin 0in
    \raggedright
    }
    \item
  }
  {\end{list}}

% another environment that's an indented list, with no spaces between items -- if we want multiple items/lines. Useful in tables. Use \item inside the environment.
% 	\raggedright makes them left aligned instead of justified
\newenvironment{indentlist}{
    \begin{list}{}{ \itemsep 0in \itemindent 0in
    \labelsep 0in \labelwidth 0in
    \listparindent 0in
    \topsep 0in \partopsep 0in \parskip 0in \parsep 0in
    \leftmargin 1em \rightmargin 0in
    \raggedright
    }

  }
  {\end{list}}



%%%%%%%%%%%%%%%%%%%%%%%%%%%%%%%%%%%%%%%%%%%%%%%%%%%%%%%%%%%%%\
%%%% Front-matter

% For early drafts, you may want to disable some of the frontmatter. Simply change this to "\ifodd 1" to do so.
\ifodd 0
% front-matter disabled while writing chapters
\renewcommand{\maketitlepage}{}
\renewcommand*{\makeauthorshipstatement}{}
\renewcommand*{\makeabstract}{}

% you can just skip the \acknowledgements and \dedication commands to leave out these sections.

\else

\abstract{
% Abstract can be any length, but should be max 350 words for a Dissertation for ProQuest's print indicies (150 words for a Master's Thesis) or it will be truncated for those uses.
In this paper, I will find features for authentication in the existing 3-D palmprint database. The features shall be stable in samples of a single person over time and distinguishable among samples of different people.

An identification process based on the features will be proposed.

The process will go through 8,000 samples in the database to be checked for stability and error rate.
}

\acknowledgements{
%I would like to thank...
%!TEX root = thesis.tex

Firstly, I would like to thank my advisor, David Zhang, for leading me into biometrics research, for insightful remarks and useful advice during my MSc. study, and for his advice on how to conduct scientific research. It has been an privilege to work under his supervision.

I would also like to thank Professor Lei Zhang for what I have learned in the Multimedia Computing course. The experience not only leads to a more systematic knowledge of the area, but also urges me to practice everything with real data.

During my time working on this dissertation, Dr. Wei Li helped me a lot in the data processing part. I want to thank him for saving me a huge amount of time dealing with the dataset.
}

\dedication{To my parents.}

\fi  % disable frontmatter


%%%%%%%%%%%%%%%%%%%%%%%%%%%%%%%%%%%%%%%%%%%%%%%%%%%%%%%%%%%%%\
%%%% Hide some chapters

%%% If you want to produce a pdf that includes only certain chapters, specify them with includeonly, in addition to including all chapters below.
%\includeonly{ch-intro/chapter-intro}
%%% You can also specify multiple chapters.
%\includeonly{ch-intro/chapter-intro,ch-usage/chapter-usage}
%\includeonly{chap1,chap2,chap3}


%%%%%%%%%%%%%%%%%%%%%%%%%%%%%%%%%%%%%%%%%%%%%%%%%%%%%%%%%%%%%
%%%% Notes:

% Footnotes should be placed after punctuation.\footnote{place here.}
% Generally, place citations before the period~\cite{anotherauthor}.
% The proper usage for i.e., and e.g., include commas ``(e.g., option A, option B)''

%%%%%%%%%%%%%%%%%%%%%%%%%%%%%%%%%%%%%%%%%%%%%%%%%%%%%%%%%%%%%
%%%% Import chapters
\begin{document}

\makefrontmatter
\pagestyle{fancy}

% If you've disabled frontmatter, you can insert the toc manually
%\tableofcontents\clearpage

% \include lets us split up the document (and each include starts a new page):
%!TEX root = ../thesis.tex
\chapter{Introduction\label{ch:intro}}

Palmprint has been increasingly recognized as unique and stable biometric characteristics for personal authentication. In the past decade, various methods based on two dimensional (2D) palmprint have been studied in depth. The 2D recognition techniques have proved to achieve high accuracy~\cite{Kong:2009hj}.

In recent years, three dimensional (3D) palmprint recognition devices emerge and are quite promising because of the additional depth information gathered.

Although the devices has been out for more than two years, most previous matching algorithms treat 3D information as a supplement to 2D texture images and used joint matching techniques to increase accuracy~\cite{Li:2011ur, Li:2010en, Zhang:2009dp, Zhang:2008kc, Zhang:2010uu}. Authentication with only the 3D information has not been thoroughly studied. The amount of useful information carried by the 3D data is still under investigation.

There are two major challenges:

\begin{enumerate}
\item 3D devices, compared to 2D ones, are lower in resolution.
\item The depth values are susceptible to movements of human hands and are therefore less stable than 2D texture information of palmprints.
\end{enumerate}

David et al. explore a 3D palmprint recognition approach by exploiting the 3D structural information of the palm surface~\cite{Zhang:2009dp, Zhang:2008kc}. The structured light imaging is used to acquire the 3D palmprint data, from which several types of unique features, including mean curvature image, Gaussian curvature image, and surface type, are extracted. A fast feature matching and score-level fusion strategy is proposed for palmprint matching and classification. Wei et al. propose an efficient joint 2D and 3D palmprint matching scheme~\cite{Li:2010en}. To align the samples accurately, features of lines and palm shape are extracted. A number of rules are then defined for matching the features efficiently for recognition purpose. Experiments show that the 3D features are not correlated with existing 2D ones and palmprint verification performance can be greatly improved by using both. Wei et al. also present an efficient scheme for 3D palmprint recognition~\cite{Li:2011ur}. They extract both line and orientation features by calculating and enhancing Gaussian-curvature and mean-curvature image or the 3D palmprint sample. A fusion of both types of features or scores is used for the 3D palmprint recognition experiments.

Existing work has been done to utilize the 3D information for palmprint classification and sorting. The global features proposed for that purpose are fast in matching speed but low in accuracy compared to 2D techniques.

Three features are extracted from 3D palmprint depth data: Maximum Depth, Horizontal Cross-section Area and Radial Line Length. Theses features are combined together as a brief description of a 3D palmprint sample. Viewing the feature set as a column vector, various data processing techniques can be applied to improve the recognition efficiency such as dimension reduction and machine learning.

The rest of this work is organized as follows. Chapter \ref{ch:pastwork} briefly introduces the existing research efforts related to this topic. Chapter \ref{ch:methodology} describes the entire data processing procedures to extract features and use them for 3D palmprint recognition. Chapter \ref{ch:experiment} shows some experiment results. Chapter \ref{ch:conclusion} concludes the work and points out potentials for some future work.


\chapter{Related Work\label{ch:pastwork}}

This work is based on a number of previous research efforts.

% include other files for sections of this chapter. These use the 'input' command since each section within a chapter should not start a new page.
% If you want to swap the order of sections, it is as simple as reversing the order you include them. 
\section{Hardware}
\label{sec:pastwork:hardware}

% ~\cite{Li:2011ur,Li:2010en,Zhang:2009dp}

Different approaches are available for 3D imaging including laser scanning~\cite{Blais:1988te}, viewpoint reconstruction~\cite{Hartley:2000un} and structured light scanning~\cite{Halioua:1984ue}. Structured light scanning, compared to other approaches, is fast and accurate. For palmprint recognition, speed is an important factor. The verification or identification result must be given in a short time. Otherwise the system is barely suitable for real-world applications.

% http://en.wikipedia.org/wiki/Structured-light_3D_scanner

Projecting a narrow band of light onto a three-dimensionally shaped surface produces a line of illumination that appears distorted from other perspectives than that of the projector, and can be used for an exact geometric reconstruction of the surface shape (light section).

A faster and more versatile method is the projection of patterns consisting of many stripes at once, or of arbitrary fringes, as this allows for the acquisition of a multitude of samples simultaneously. Seen from different viewpoints, the pattern appears geometrically distorted due to the surface shape of the object.

\begin{figure}[htb]
  \begin{center}
    \includegraphics[width=0.9\linewidth]{ch-pastwork/figures/sli}
    \caption[The principle of structured-light imaging]{The principle of structured-light imaging\cite{Li:2009eq}}
    \label{fig:pastwork:sli}
  \end{center}
\end{figure}

Although many other variants of structured light projection are possible, patterns of parallel stripes are widely used. Figure~\ref{fig:pastwork:strippattern} shows the geometrical deformation of a strip pattern projected onto a palm. The displacement of the stripes allows for an exact retrieval of the 3D coordinates of any details on the palm's surface.

David et. al designed a 3D palmprint capturing device~\cite{Zhang:2008kc}. The system proposed has a resolution of 768x576.

\begin{figure}[htb]
  \begin{center}
    \includegraphics[width=0.9\linewidth]{ch-pastwork/figures/strippattern}
    \caption[Sample patterns of the stripes on the palm]{Sample patterns of the stripes on the palm~\cite{Li:2009eq}}
    \label{fig:pastwork:strippattern}
  \end{center}
\end{figure}

%!TEX root = chapter-pastwork.tex
\section{Recognition Methods}
\label{sec:pastwork:recogmethods}

Traditionally, palmprint recognition has made use of either high or low resolution 2D palmprint images. High resolution images are suitable for forensic applications ~\cite{Jain:2009kj} while low resolution images are suitable for civil and commercial applications ~\cite{Zhang:2003uf}. Most current research use low resolution palmprint recognition and is either texture-based or line-based. The texture-based methods include PalmCode ~\cite{Zhang:2003uf}, Competitive Code ~\cite{Kong:2004wq} and Ordinal Code ~\cite{Zhenan:2005wg}. These methods use a group of filters to enhance and extract the phase or directional features which can represent the texture of the palmprint. Line-based methods use line or edge detectors to explicitly extract line information from the palmprint that is then used for matching. The representative methods include Derivative of Gaussian based line extraction ~\cite{Wu:2006wa} and Modified Finite Radon transform (MFRAT) based line extraction ~\cite{Huang:2008ep}.

In recent years, 3D techniques have been applied to biometric authentication, such as 3D face ~\cite{Samir:2006vj} and 3D ear recognition ~\cite{Yan:2007fv}. Most recently, a structured-light imaging ~\cite{Halioua:1984ue,Saldner:1997tu} 3D palmprint system ~\cite{Zhang:2009dp} was developed that captures the depth information of a palmprint. This information is then used to calculate the Mean and the Gaussian curvatures for use in 3D palmprint matching and recognition. To date, however, there has been no work with 3D palmprints that has extracted global shape features, which may be useful in classification and indexing. For fingerprint, according to the global ridge structure and singularities, it can be classified into five classes: arch, tented arch, left loop, right loop and whorl ~\cite{Henry:1900vc}. Wu et al. classified the palmprint into six classes according to the palmprint principal lines ~\cite{Wu:2004kx}. Besides the exclusive classification technique, the continuous classification technique is also widely used for indexing the database for personal identification ~\cite{Lumini:1997vv}.

Some general features were extracted for recognition including 3D Mean Curvature Image, 3D Gauss Curvature Image and 3D Surface Type~\cite{Zhang:2008kc,Li:2009eq}.





\chapter{Methodology\label{ch:methodology}}

The idea is to extract one or more features from the 3-D information as classifiers. Together with existing classifiers found in existing work in related fields~\cite{Xu:2004tt,Yan:2007fv,Samir:2006vj}, the new features will be combined to achieve a high performance classifier for personal authentication using Support Vector Machine (SVM).

The sample data has already been collected. There are 8,000 samples from 400 different palms with both hands. Samples of a palm are collected in two separate sessions with an average interval of one month.

Experiments will be done using Matlab.

1. Extract features from each palmprint sample.

2. A subset of samples will be chosen as test set.

3. Train an authentication model based on the rest of samples.

4. Verify the samples in the test set with the trained model against their true identities.

5. Discuss the performance of the model.

\section{Region of Interest Extraction}
\label{sec:methodology:roiextraction}

\section{Feature Calculation}
\label{sec:methodology:featurecalc}

\section{Feature Matching}
\label{sec:methodology:featurematch}

The classification of biometrics speeds up the identification process by reducing the number of comparisons that must be made. There are two kinds of classification techniques: exclusive classification and continuous classification. Both fingerprint [14] and palmprint classifications [15] make use of exclusive classification. The main problem of this technique is that it uses only a small number of classes and the samples are unevenly distributed between them, with more than 90\% of the samples being in just two or three classes. A further problem with exclusive classification is that when classification is performed automatically, it is necessary to handle errors and rejected samples gracefully, which is a hard problem in practice. In contrast, for continuous classification, samples are not partitioned into disjoint classes but rather associated with numerical vectors which represent features of the samples. These feature vectors are created through a similarity-preserving transformation so that similar samples are mapped into close points in the multi-dimensional space [19]. In this paper, we adopt the continuous classification technique. As the global features combining MD, HCA and RLL are high-dimensional, we reduce the dimensions using the LDA method. We then improve the efficiency of palmprint recognition by applying coarse-level matching and Ranking Support Vector Machine (RSVM) to the low dimensional vectors.

%!TEX root = featurematch.tex
\subsection{Dimension Reduction}
\label{ssec:methodology:lda}

%TODO find the citation
LDA is a state-of-the-art dimensionality reduction technique widely used in classification problems. The objective is to find the optimal projection which simultaneously minimizes the within-class distance and maximizes the between-class distance, thus achieving maximum discrimination (Here, the “class” is used to denote the identity of the subjects, e.g. the samples collected from one palm are regarded as one class). However, the traditional LDA requires the within-class scatter matrix to be nonsingular, which means the sample size should be large enough compared with its dimension, but is not always possible. In this paper, we therefore adopt the orthogonal LDA (OLDA) proposed in ~\cite{Ye:2005uu}, where the vectors of the optimal projection are calculated using the training database and the optimal projecting vectors are orthogonal to each other.

Suppose the 3D ROI has been divided to N levels and that M radial lines are used to represent the level contours. We can list the features as a column vector,
\begin{equation}
F=\{MD,A^1,A^2,\dots,A^N,R^1,R^2,\dots,R^{N\times M}\}
\end{equation}
with $1+N+N\times M$ rows. Given a training database which has $n$ samples and $k$ classes as $X=[X_1,X_2,\dots,X_k]$, where $X_i \in \mathbb{R}^{(1+N+N\times M)\times n}, i=1,2,\dots,k$ and $n=\sum \limits_{i=1}^k n_i$ , adopting OLDA ~\cite{Ye:2005uu} the optimal projection $W$ can be calculated as follows.

First, the within-class scatter matrix $S_w$, the between-class scatter matrix $S_b$ and total scatter matrix $S_t$ can be expressed as

\begin{equation}
S_w=H_w H_w^T, S_b=H_b H_b^T, S_t=H_t H_t^T
\end{equation}

where

\begin{equation}
H_w=\frac{1}{\sqrt{n}}\left[X_1 - m_1 \cdot e_1^T, \dots, X_k - m_k \cdot e_k^T \right]
\end{equation}

\begin{equation}
H_b=\frac{1}{\sqrt{n}}\left[\sqrt{n_1}(m_1-M),\dots,\sqrt{n_k}(m_k-m) \right]
\end{equation}

\begin{equation}
H_t=\frac{1}{\sqrt{n}}(X-m\cdot e^T)
\end{equation}

                                         (9)
where $m_i$ is the centroid of the $i$th class $X_i$, $m$ is the centroid of all the training samples $X$, $e_i=[1,1,\dots,1]^T \in \mathbb{R}^n, i=1,2,\dots,k$ and $e=[1,1,\dots,1]^T \in \mathbb{R}^n$.

After calculating $H_w$, $H_b$ and $H_t$, the reduced Singular Value Decomposition (SVD) is applied to $H_t$.

\begin{equation}
H_t \xrightarrow{\text{Reduced SVD}} U_r \Sigma_r V_r^T
\end{equation}

Denote $B=\Sigma_r^{-1} U_r^T H_b$ and compute the SVD of B.

\begin{equation}
B \xrightarrow{\text{SVD}} U_B \Sigma_B V_B^T
\end{equation}

Let

\begin{equation}
D=U_r\Sigma_r^{-1}U_B
\end{equation}

\begin{equation}
q=rank(B)
\end{equation}

and denote $D_q$ the first $q$ columns of the matrix D. Then, compute the QR decomposition of $D_q$.

\begin{equation}
D_q \xrightarrow{\text{QR decomposition}} QR
\end{equation}

where $Q$ is the desired orthogonal matrix and optimal projection, i.e. $W=Q$.

After getting the optimal projection $W$, we can map the $1+N+N\times M$ dimensional vector $F$ to a lower dimensional space

\begin{equation}
\tilde{F}=W^T F
\end{equation}
                                             (15)
where $\tilde{F}={f_1,f_2,\dots,f_{\Gamma}}$ is a $\Gamma$ dimensional vector with $\Gamma<1+N+N\times M$.

\subsection{Naive Matching}
\label{ssec:methodology:naive}

As the purpose of coarse-level matching is to speed up the identification during retrieval, we can regard it as a continuous classification approach. After mapping the global features’   dimensional vector  , the global features can be used to measure the similarity of two samples as follows:
                              (16)
In 3D palmprint identification, we can use the   dimensional global features to carry out coarse-level matching as shown in Fig. 9. If the testing sample passes coarse-level matching, it undergoes fine-level matching using 3D palmprint local features. If it does not pass, it moves on to the next sample in the database and so on until it has accessed the last sample in the database. From (16), we can see that coarse-level matching requires only   times of addition and multiplication which is much faster than fine-level matching using local features. Eq. (17) gives the fine-level matching by Mean Curvature Image (MCI) feature [13]:
                              (17)
where symbol “ ” represents the logical AND operation,   and   are the two binarized MCI features. To deal with the translation problem of ROI when calculating the matching score by (17), we will shift two, four, six and eight pixels of the test image along 8 directions: right, left, up, down, left-up, left-down, right-up and right-down, respectively. Adding the non-shift one, we will have   matching scores and the maximum one is selected. Suppose the size of the MCI feature is  , i.e.   and  , from (16) and (17) we can see that coarse-level matching is much faster than fine-level matching.

\subsection{Support Vector Machine}
\label{ssec:methodology:svm}

Coarse-level matching scheme is a simple and easy way to reduce retrieval times. It’s more useful for palmprint recognition if we can rank the candidate samples in the database in descending order according to the above global features. Searching for the closest matches to a given query vector in a large database is time consuming if the vector is even moderately high-dimensional. Various methods have been proposed to speed up the nearest neighbor retrieval, including hashing and tree structures [20]. However, the complexity of these methods grows exponentially with increasing dimensionality [21]. Therefore, we have adopted the Ranking Support Vector Machine (RSVM) method [18], inspired by the approaches of internet search engines, to rank the candidate samples in the database.

Figure 9. The flowchart of registration and recognition with coarse-level matching scheme.

Given a query   and a sample collection , the optimal retrieval system should return a ranking   that orders the samples in D according to their relevance to the query. In this paper, the query   and the sample   are the   dimensional global features as described above. In our approach, if a sample   is ranked higher than   in some ordering  , i.e.  , then  , otherwise  . Consider the class of linear ranking functions
                           (18)
where   is a weighted vector that is adjusted by learning and   is a pairwise distance function describing the match between   and   can be defined as  . Our goal is to find the optimal ranking function that will satisfy the maximum number of the following inequalities.
                                (19)
It is easier to solve this problem if it is converted into to the following SVM classification problem by introducing non-negative slack variable  .
Hence, the problem is to minimize:
                                   (20)
subject to:
                       (21)
                                              (22)
where C is a parameter that allows trading-off margin size against training error and   set by experience.
In the training stage, it is the inner-class samples of a test sample that should be ranked higher than the inter-class samples, e.g. inner-class samples rank is 1 and inter-class samples rank is 0. We input the ranks together with the   dimensional global features into the RSVM algorithm to learn the optimal ranking function  . Given a new query  , the samples in the database can be sorted by their value of
                                  (23)



\chapter{Experimental Results\label{ch:experiment}}

\section{Feature Distribution}
\label{sec:experiment:featuredist}

\section{Identificaiton Performance}
\label{sec:experiment:identificaiton}

Table 3 and Fig. 10 show the recognition results. We can see that 15 dimensions is a good choice for the following coarse-level matching and RSVM schemes.

Fig. 11 shows the genuine and imposter distributions when the 3D palmprint 15-dimensional global features are applied to the 4000 samples in the training database. Figs. 11 (a) to (o) are obtained by using the Euclidian distance to match the single dimension value from the 1st to 15th. Fig. 11 (p) shows the result of using the Euclidian distance to match all 15 dimensional values.
We next carried out the 3D palmprint classification and recognition experiments using the first sample of each class in the training database as a template and the 4000 samples in the testing database as probes, making a total of 400 templates and 4000 probes. The performance of classification and recognition is usually measured by error rate and penetration rate calculated in [19] as follows:
                               (24)
           (25)


         (a)                   (b)                    (c)                   (d)

         (e)                    (f)                   (g)                   (h)

          (i)                   (j)                    (k)                  (l)

          (m)                   (n)                   (o)                   (p)
Figure 11. Genuine and imposter distributions by the 3D palmprint 15-dimensional global features. (a) to (o) are obtained by matching the single dimension value from the 1st to 15th and (p) is obtained by matching all 15 dimensional values together.

Obviously there is a trade-off between error rates and penetration rates. Generally speaking, if there is no classification, there are two retrieval strategies: 1) all of the templates in the database are visited and the template that gives the best matching score is regarded as the matched template, if the matching score is less than a given threshold  ; 2) given a threshold  , the search continues until a match is found that is below that threshold.
We used three 3D palmprint recognition matching approaches: 1) no classification; 2) coarse-level matching; and 3) RSVM. For no classification, we matched using the local feature MCI as described in [13]. The process we used for coarse-level matching is illustrated in Fig. 9 and involves fine-level matching using the local feature MCI. A single instance of coarse-level matching requires only 1/36000 of the time it takes to do fine-level matching (coarse-level matching only needs 15 operations while fine-level matching must do   operations, where   is the size of ROI and   is the shifting template times). For the above two approaches, the penetration rate and the error rate will vary with different thresholds  . As for RSVM, we use the RSVM algorithm described in section 3.3 to rank the templates in the database, and then match the top   percent by local feature MCI with the best matching score regarded as the matched template if this score is less than a given constant threshold  . We can see from (25) that the   is equal to the penetration rate. Given different thresholds   and  , we carried out a series of 3D palmprint recognition experiments. Table 4, Table 5 and Fig. 12 show these experimental results. Even at an approximately equal error rate, the proposed coarse-level matching and RSVM approaches get a much lower penetration rate than the no classification approach. Obviously RSVM has the best performance but requires an additional offline training process compared to coarse-level matching.

% Table generated by Excel2LaTeX from sheet 'Sheet2'
\begin{table}[htbp]
\centering
\caption{EER of 3D palmprint verification with different feature sets}
\begin{tabular}{|c|c|c|c|c|}
\hline
Global features & MD    & HCA   & RLL   & MD+HCA+RLL \\
\hline
EER(\%) & 25.8  & 20.4  & 18.6  & 12.32 \\
\hline
\end{tabular}%
\label{tab:experiment:verification}%
\end{table}%


\chapter{Conclusions\label{ch:conclusion}}

In this chapter we conclude this dissertation by summarizing our contributions and discussing directions for future work.

\section{Summary}
% In this work, we explain how to use the puthesis.cls class file and the accompanying template.

This paper proposed three global features for 3D palmprint images: Maximum Depth (MD), Horizontal Cross-section Area (HCA) and Radial Line Length (RLL). These cannot be extracted from 2D palmprints and are not correlated with local features, such as line and texture features. To make these global features efficient for use in coarse classification, we treat them as a multi-dimensional vector and use OLDA to map it to a lower dimensional space. We then improve the efficiency of 3D palmprint recognition using two proposed approaches, coarse-level matching and RSVM, both of which significantly reduce the penetration rate during retrieval. Our recognition experiments using an established 3D palmprint database of 8,000 samples show that the global features improve palmprint classification which greatly reduces search times. % Conclusion
%!TEX root = chapter-conclusion.tex
\section{Future Work}

As shown in Chapter ~\ref{ch:experiment}, the error rate in recognition using only the (Maximum Depth, Horizontal Cross-section Area, Radial Line Length are much higher than the cases when Mean Curvature Image is used. The problem with the latter one is its computation time.

There has not been a thorough analysis of possible features of the pure depth map. It remains unknown how much performance gain can be obtained by incorporating the 3D system.
 % Future work



% add the Bibliography to the Table of Contents
\cleardoublepage
\ifdefined\phantomsection
   \phantomsection  % makes hyperref recognize this section properly for pdf link
\else
\fi
\addcontentsline{toc}{chapter}{\bibname}

% include your .bib file
\bibliography{thesis}

\bibliographystyle{unsrt}

\appendix % all chapters following will be labeled as appendices
%!TEX root = ../thesis.tex
\chapter{Core Matlab Code\label{ch:code}}

\singlespacing

Create large arrays without asking for contiguous memory.

\begin{minted}[fontsize=\small,frame=lines,linenos]{matlab}
function result = createArrays(nArrays, arraySize, datatype)
%CREATEARRAYS
% This creates an cell array which does NOT require
% contiguous memory.
%
% To use it:
%   myArray = createArrays(numberOfArrays, [x y], 'single');
%
% To access the elements:
%   myArray{1}{2,3} = 10;
%   myArray{1} = zeros(500, 800, 'single');
    result = cell(1, nArrays);
    for i = 1 : nArrays
        result{i} = zeros(arraySize, datatype);
    end
end
\end{minted}

\clearpage

Code to find the reference plane.

\begin{minted}[fontsize=\small,frame=lines,linenos]{matlab}
function [ d_ref ] = find_ref_plane( roi )
%FIND_REF_PLANE Find the depth of reference plane of an ROI
%   Detailed explanation goes here

	[~,roi_size] = size(roi);

	refplane = roi(1:(roi_size*0.25),(roi_size*0.25):(roi_size*0.9));

	d_ref = max(max(refplane));

end
\end{minted}

\clearpage

Code to extract the X coordinates from original data.
\begin{minted}[fontsize=\small,frame=lines,linenos]{matlab}
clear;

prefix = '..\..\3D_palm';

%% File to process
load_dir = [prefix filesep '3DPalm_xyz'];
file_mask = [load_dir filesep '*.dat'];
file_list = dir(file_mask);
file_list = char({file_list.name});

%% Get X coordinates from file
tic;
dat_file = reshape(dlmread([load_dir filesep file_list(1,:)]),768,576,3);
palmX=dat_file(:,1,1);

%% Save result to file
save_dir = ['output'];
save([save_dir filesep '3Dpalm_x.mat'],'palmX');
toc;
\end{minted}

\clearpage

Code to remove outliers from a vector.

\begin{minted}[fontsize=\small,frame=lines,linenos]{matlab}
function [b,idx,outliers] = deleteoutliers(a,alpha,rep);
% [B, IDX, OUTLIERS] = DELETEOUTLIERS(A, ALPHA, REP)
% 
% For input vector A, returns a vector B with outliers (at the
% significance level alpha) removed. Also, optional output 
% argument idx returns the indices in A of outlier values. Optional
% output argument outliers returns the outlying values in A.
%
% ALPHA is the significance level for determination of outliers.
% If not provided, alpha defaults to 0.05.
% 
% REP is an optional argument that forces the replacement of
% removed elements with NaNs to presereve the length of a.
%
% This is an iterative implementation of the Grubbs Test that
% tests one value at a time. In any given iteration, the tested 
% value is either thehighest value, or the lowest, and is the
% value that is furthest from the sample mean. Infinite elements
% are discarded if rep is 0, or replaced with NaNs if rep is 1.
% 
% Appropriate application of the test requires that data can be
% reasonably approximated by a normal distribution. For reference,
% see:
% 1) "Procedures for Detecting Outlying Observations in Samples,"
%     by F.E. Grubbs; Technometrics, 11-1:1--21; Feb., 1969, and 
% 2) _Outliers in Statistical Data_, by V. Barnett and
%    T. Lewis; Wiley Series in Probability and Mathematical
%    Statistics; John Wiley & Sons; Chichester, 1994.
% A good online discussion of the test is also given in NIST's
% Engineering Statistics Handbook:
% http://www.itl.nist.gov/div898/handbook/eda/section3/eda35h.htm

if nargin == 1
	alpha = 0.05;
	rep = 0;
elseif nargin == 2
	rep = 0;
elseif nargin == 3
	if ~ismember(rep,[0 1])
		error('Please enter a 1 or a 0 for optional argument rep.')
	end
elseif nargin > 3
	error('Requires 1,2, or 3 input arguments.');
end

if isempty(alpha)
	alpha = 0.05;
end

b = a;
b(isinf(a)) = NaN;

%Delete outliers:
outlier = 1;
while outlier
	tmp = b(~isnan(b));
	meanval = mean(tmp);
	maxval = tmp(find(abs(tmp-mean(tmp))==max(abs(tmp-mean(tmp)))));
	maxval = maxval(1);
	sdval = std(tmp);
	tn = abs((maxval-meanval)/sdval);
	critval = zcritical(alpha,length(tmp));
	outlier = tn > critval;
	if outlier
		tmp = find(a == maxval);
		b(tmp) = NaN;
	end
end
if nargout >= 2
	idx = find(isnan(b));
end
if nargout > 2
	outliers = a(idx);
end
if ~rep
	b=b(~isnan(b));
end
return

function zcrit = zcritical(alpha,n)
%ZCRIT = ZCRITICAL(ALPHA,N)
% Computes the critical z value for rejecting outliers (GRUBBS TEST)
tcrit = tinv(alpha/(2*n),n-2);
zcrit = (n-1)/sqrt(n)*(sqrt(tcrit^2/(n-2+tcrit^2)));
\end{minted}
\clearpage

Code for batch feature extraction.

\begin{minted}[fontsize=\small,frame=lines,linenos]{matlab}
clear;

%% Files to process
save_dir = ['output'];

use_original_data = true;
% Toggle this for data source selection
% use_original_data = false;

if (use_original_data)
    data_dir = ['..' filesep 'rawdata'];
    file_mask = [data_dir filesep '*.zonly'];
    sample_width = 768;
    sample_height = 576;
    roi_size = 400;
else
    data_dir = ['..' filesep 'rawdata' filesep 'roi'];
    file_mask = [data_dir filesep '*.dat'];
    sample_width = 128;
    sample_height = 128;
    roi_size = 128;
end

file_list = dir(file_mask);
file_list = char({file_list.name});

num_of_samples = length(file_list);
sample_per_person = 10;
num_of_people = num_of_samples/sample_per_person;
% Toggle this for limited number of files
// num_of_people = 100;

feature_dimension = roi_size;
% feature_dimension = roi_size*roi_size;

features = zeros(num_of_people*sample_per_person,feature_dimension);

%% Parallel read files
tic;
for current_person = 1:num_of_people
    disp(['Loading data from ' num2str(current_person) ' of \
		' num2str(num_of_people) ' people.']);
    for current_sample = 1:sample_per_person
        file_id = (current_person - 1) * 10 + current_sample;
        sample_filename = strtrim([data_dir filesep file_list(file_id,:)]);
        
        mat = file2matrix(sample_filename, sample_width, sample_height);
        
        if (use_original_data)
            % Crop ROI from original data
            mat = mat(235:(234+roi_size),69:(68+roi_size));
        end;
        
        % Extract feature for this sample
        features(file_id,:) = calc_feature(mat);
    end
end

save(['output' filesep 'features.mat'], 'features');
toc;
\end{minted}
\clearpage

Code for curvature extraction.

\begin{minted}[fontsize=\small,frame=lines,linenos]{matlab}
function [K,H,Pmax,Pmin] = surfature(X,Y,Z),
% SURFATURE -  COMPUTE GAUSSIAN AND MEAN CURVATURES OF A SURFACE
%   [K,H] = SURFATURE(X,Y,Z), WHERE X,Y,Z ARE 2D ARRAYS OF POINTS
%   ON THE SURFACE.  K AND H ARE THE GAUSSIAN AND MEAN CURVATURES,
%   RESPECTIVELY.
%
%   SURFATURE RETURNS 2 ADDITIONAL ARGUEMENTS,
%
%   [K,H,Pmax,Pmin] = SURFATURE(...), WHERE Pmax AND Pmin ARE THE
%   MINIMUM AND MAXIMUM CURVATURES AT EACH POINT, RESPECTIVELY.


% First Derivatives
[Xu,Xv] = gradient(X);
[Yu,Yv] = gradient(Y);
[Zu,Zv] = gradient(Z);

% Second Derivatives
[Xuu,Xuv] = gradient(Xu);
[Yuu,Yuv] = gradient(Yu);
[Zuu,Zuv] = gradient(Zu);

[Xuv,Xvv] = gradient(Xv);
[Yuv,Yvv] = gradient(Yv);
[Zuv,Zvv] = gradient(Zv);

% Reshape 2D Arrays into Vectors
Xu = Xu(:);   Yu = Yu(:);   Zu = Zu(:); 
Xv = Xv(:);   Yv = Yv(:);   Zv = Zv(:); 
Xuu = Xuu(:); Yuu = Yuu(:); Zuu = Zuu(:); 
Xuv = Xuv(:); Yuv = Yuv(:); Zuv = Zuv(:); 
Xvv = Xvv(:); Yvv = Yvv(:); Zvv = Zvv(:); 

Xu          =   [Xu Yu Zu];
Xv          =   [Xv Yv Zv];
Xuu         =   [Xuu Yuu Zuu];
Xuv         =   [Xuv Yuv Zuv];
Xvv         =   [Xvv Yvv Zvv];

% First fundamental Coeffecients of the surface (E,F,G)
E           =   dot(Xu,Xu,2);
F           =   dot(Xu,Xv,2);
G           =   dot(Xv,Xv,2);

m           =   cross(Xu,Xv,2);
p           =   sqrt(dot(m,m,2));
n           =   m./[p p p]; 

% Second fundamental Coeffecients of the surface (L,M,N)
L           =   dot(Xuu,n,2);
M           =   dot(Xuv,n,2);
N           =   dot(Xvv,n,2);

[s,t] = size(Z);

% Gaussian Curvature
K = (L.*N - M.^2)./(E.*G - F.^2);
K = reshape(K,s,t);

% Mean Curvature
H = (E.*N + G.*L - 2.*F.*M)./(2*(E.*G - F.^2));
H = reshape(H,s,t);

% Principal Curvatures
Pmax = H + sqrt(H.^2 - K);
Pmin = H - sqrt(H.^2 - K);
\end{minted}
\clearpage

Code to test feature performance.

\begin{minted}[fontsize=\small,frame=lines,linenos]{matlab}
%% Load features
load(['output' filesep 'features.mat']);

[total,~] = size(features);

%% Split into training set and testing set
training_id=unique([
    1:10:total
    2:10:total
    3:10:total
    4:10:total
    5:10:total
    6:10:total
    ]);

trainingset = features(training_id,:);

test_id=unique([
    7:10:total
    8:10:total
    9:10:total
    10:10:total
    ]);

testingset = features(test_id,:);

%% Find match for each test input
nearest_id = knnsearch(trainingset, testingset, 'k', 10);

%% Interpret the match to person
[total_training,~] = size(trainingset);
training_sample_per_person = total_training / total * 10;
nearest_person = ceil(nearest_id/training_sample_per_person);

%% Check performance
[total_testing,~] = size(testingset);
testing_sample_per_person = total_testing / total * 10;
for i = 1:total_testing
    correct(i) = eq(nearest_person(i),ceil(i/testing_sample_per_person));
end
correct = correct';

accuracy = sum(correct)/total_testing
\end{minted}
\clearpage

Code to extract shape features.

\begin{minted}[fontsize=\small,frame=lines,linenos]{matlab}
function extractShapeF()

sizeH = 200; %
sizeW = 200; %
lnum = 8;

pixDis = 0.28;
% load facesMatrix64.mat;
temp = pixDis * ((sizeH-2)/2+0.5);
[X,Y] = meshgrid(-temp : pixDis : temp);

fid = fopen('../rawdata/Sub3D_I_3_0.dat', 'r');
% fid = fopen('../rawdata/Sub3D_II_100_0.dat', 'r');
Z = fread(fid, [sizeH,sizeW], 'double');
fclose(fid);
[fx, fy] = gradient(Z);
fxy = fx.^2 + fy.^2;
noiseP = find(fxy>0.1); % 0.1 used for corrected and smoothed Sub3D, 1 used for original Sub3D
flag = ones(sizeH, sizeW); %1 for valid point, 0 for invalid point
flag(noiseP) = 0;
flag = 1 - flag;
se = strel('disk',5);  
flag = imdilate(flag, se);
flag = 1 - flag;
noiseP = find(flag==0);
meanZ = sum(sum(Z .* flag)) / (sizeH*sizeW - length(noiseP));
Z(noiseP) = 5;
%neend't smooth
% Z = smooth(Z, 7);

%%for show the mask which get rid of bad quality region
% flag(1,:) = 0; flag(end,:) = 0;
% flag(:,1) = 0; flag(:,end) = 0;
% imshow(flag');

%calculate the reference 0 plane
% use Z(6:35, 65:136) for calculate the mean value as reference 0
refRect = Z(6:35, 65:136);
refFlag = flag(6:35, 65:136);
refVal = sum(sum(refRect .* refFlag)) / sum(sum(refFlag));
Z = Z - refVal;

%search the min point in Z(65:190, 41:160)
rectforMin = Z(65:190, 41:160);
rectforMin(1:35, 1:25) = 5;
rectforMin(1:35, end-25:end) = 5;
% flagforMin = Z(65:190, 41:160);
palmH = min(min(rectforMin));
ind = find(Z == palmH);
% Z(ind) = 5;

% %find the level regions from 0 to deepest point 
% %find the region > 0
% palmH = -palmH;
% Z = -Z;
% Ln = 8;
% step = palmH / Ln;
% levelH = [0:step:palmH];
% for i = 1:Ln
%     L0 = zeros(sizeH, sizeW);
%     L0( find( Z>=levelH(Ln-i+1) ) ) = 1;
%     %the 1st level
%     if i==1
%         [L,num] = bwlabel(L0);
%         if num > 1
%             for j = 1:num
%                 indL = find(L==j);
%                 if length(find(indL==ind)) > 0
%                     L0(:,:) = 0;
%                     L0(indL) = 1;
%                 end
%             end
%         end
%         Lp = L0;
%         L0 = logical(L0');  
%         [x, y] = find(L0);
%         temp = find(L0);
%         mx = mean(x);
%         my = mean(y);
% 
% %         saveIm(L0, i);
%         figure;
%         imshow(L0);
%     else
%         %dilate the Lp (previous level) and than & with L0
%         se = strel('disk',35 - 3*i);  
%         L1 = imdilate(Lp, se);
%         L0 = L0 & L1;
%         Lp = L0;
%         L0 = L0';   
% %         saveIm(L0, i);
%         figure;
%         imshow(L0);
%     end    
% end

%for show
im_level = zeros(sizeH, sizeW); %for show the levels
%find the level regions from 0 to deepest point 
%find the region > 0
palmH = -palmH;
Z = -Z';
Ln = 8;
step = palmH / Ln;
levelH = [0:step:palmH];
for i = 1:Ln
    L0 = zeros(sizeH, sizeW);
    L0( find( Z>=levelH(Ln-i+1) ) ) = 1;
    %the 1st level
    if i==1
        [L,num] = bwlabel(L0);
        if num > 1
            for j = 1:num
                indL = find(L==j);
                if length(find(indL==ind)) > 0
                    L0(:,:) = 0;
                    L0(indL) = 1;
                end
            end
        end
        Lp = L0;
        [x, y] = find(L0);
        temp = find(L0);
        mx = round(mean(x));
        my = round(mean(y));
        im_level(temp) = 155;

%         saveIm(L0, i);
%         figure;
%         imshow(L0);
    else
        %dilate the Lp (previous level) and than & with L0
        se = strel('disk',35 - 3*i);  
        L1 = imdilate(Lp, se);
%         figure; imshow(Lp);
%         figure; imshow(L1);
        L0 = L0 & L1;
%         figure; imshow(L0);
        Ltemp = L0 - Lp;
%         figure; imshow(Ltemp);
        Lp = L0; 
        temp = find(Ltemp);
        im_level(temp) = 255-(i-1)*30;
%         saveIm(L0, i);
%         figure;
%         imshow(L0);
    end    
end

im_level = uint8(im_level);
% imshow(im_level);
% imwrite(im_level, 'figures/im_levels.bmp');

line_401 = imread('figures/line_all.bmp');
line_200 = line_401(201-mx+1:201-mx+1+199, 201-my+1:201-my+1+199);
% imshow(line_200);
temp = find(line_200);
im_level(temp) = 20;
im_level(mx, my) = 255;
figure;
imshow(im_level);
imwrite(im_level, 'figures/im_levels_line_4.bmp');
a = 0;

% %%% Z(6:35, 65:136) = 0;
% Z(6:35, 65) = 1;
% Z(6:35, 136) = 1;
% Z(6, 65:136) = 1;
% Z(35, 65:136) = 1;

% figure;
% mesh(X,Y,Z);
% daspect([1 1 1]);
% view([90 90]);

a = 0;

%-----------------------------------------------------------------
function saveIm(data, nlevel)
filename = ['../rawdata/bmp/Sub3D_I_4_9', '_' num2str(nlevel), '.bmp'];
imwrite(data, filename);


function [z] = smooth(z, fs)
% z -- 128*128
% n -- filter size
t = floor(fs/2)-1;
for i = 0:t;
    [m,n] = size(z);
    z = [z(:, 2*i+1) z(:,:) z(:, n-2*i)];
end

for i = 0:t;
    [m,n] = size(z);
    z = [z(2*i+1, :); z(:,:); z(m-2*i, :)];
end

h = ones(fs,fs)/(fs*fs);
z = filter2(h, z, 'valid');
\end{minted}
\clearpage
% \include{ch-appendicies/printing}

\end{document}

