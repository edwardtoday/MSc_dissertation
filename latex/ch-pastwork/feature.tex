%!TEX root = chapter-pastwork.tex
\section{Recognition Methods}
\label{sec:pastwork:recogmethods}

Traditionally, palmprint recognition has made use of either high or low resolution 2D palmprint images. High resolution images are suitable for forensic applications ~\cite{Jain:2009kj} while low resolution images are suitable for civil and commercial applications ~\cite{Zhang:2003uf}. Most current research use low resolution palmprint recognition and is either texture-based or line-based. The texture-based methods include PalmCode ~\cite{Zhang:2003uf}, Competitive Code ~\cite{Kong:2004wq} and Ordinal Code ~\cite{Zhenan:2005wg}. These methods use a group of filters to enhance and extract the phase or directional features which can represent the texture of the palmprint.

By using line detectors or edge detectors, line information can be extracted from palmprints explicitly. Line-based methods use a matching scheme with these features. The representative methods include Derivative of Gaussian based line extraction ~\cite{Wu:2006wa} and Modified Finite Radon transform (MFRAT) based line extraction ~\cite{Huang:2008ep}.

In recent years, 3D techniques have been applied to biometric authentication, such as 3D face ~\cite{Samir:2006vj} and 3D ear recognition ~\cite{Yan:2007fv}. Most recently, a structured-light imaging ~\cite{Halioua:1984ue,Saldner:1997tu} 3D palmprint system ~\cite{Zhang:2009dp} was developed that captures the depth information of a palmprint. This information is then used to calculate the Mean and the Gaussian curvatures for use in 3D palmprint matching and recognition. To date, however, there has been no work with 3D palmprints that has extracted global shape features, which may be useful in classification and indexing. For fingerprint, according to the global ridge structure and singularities, it can be classified into five classes: arch, tented arch, left loop, right loop and whorl ~\cite{Henry:1900vc}. Wu et al. classified the palmprint into six classes according to the palmprint principal lines ~\cite{Wu:2004kx}. Besides the exclusive classification technique, the continuous classification technique is also widely used for indexing the database for personal identification ~\cite{Lumini:1997vv}.

Some general features were extracted for recognition including 3D Mean Curvature Image, 3D Gauss Curvature Image and 3D Surface Type~\cite{Zhang:2008kc,Li:2009eq}.

