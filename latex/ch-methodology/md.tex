%!TEX root = featurecalc.tex
\subsection{Maximum Depth (MD)}
\label{ssec:methodology:md}

MD means the maximum depth value of the 3D palm from a reference plane. The reference plane is decided using a rectangle obtained by experience. The top left and bottom right pixels of the rectangle are (65,6) and (136,35). These parameters are denoted as $R_s=65, R_e=136, C_s=6$ and $C_e=35$, i.e. the starting(ending) row(column). By examine a random 20 samples, gradient of this area is relatively small.

The depth of the reference plane is defined as the mean depth of the points contained by this rectangle

\begin{equation}
d_r = {1\over{
    \sum \limits_{i=R_s}^{R_e} \sum \limits_{j=C_s}^{C_e} m_{ij}
}}
\sum \limits_{i=R_s}^{R_e} \sum \limits_{j=C_s}^{C_e} (d_{ij} \cdot d_{ij})
\end{equation}

where $d_{ij}$ and $m_{ij}$ are the elements defined in ~\ref{eq:methodology:roimatrix} and ~\ref{eq:methodology:roimask}

After getting the depth of the reference plane, we find the maximum depth $d_{max}$ in another region with $R_s=41, R_e=160, C_s=65$ and $C_e=190$.

\begin{equation}
d_{max} = \max \limits_{i=R_s}^{R_e} (\max \limits_{j=C_s}^{C_e} (d_{ij}) )
\end{equation}


The Maximum Depth (MD) is then defined as

\begin{equation}
\label{eq:methodology:md}
MD= d_{max} - d_r
\end{equation}