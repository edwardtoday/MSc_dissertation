%!TEX root = chapter-methodology.tex
\section{Feature Calculation}
\label{sec:methodology:featurecalc}

Using the ROI obtained from the original 3D palmprint data, three kinds of features to describe the shape of the 3D palmprint: Maximum Depth (MD) of palm center, the Horizontal Cross-section Area (HCA) of different levels and the Radial Line Length (RLL) from the centroid to the boundary of 3D palmprint horizontal cross-section of different levels.

\subsection{Maximum Depth}
\label{ssec:methodology:md}

MD means the maximum depth value of the 3D palm from a reference plane. The reference plane is decided using a rectangle as shown in the left of Fig. 3 (a). The depth of the reference plane   is the mean depth of the points contained by this rectangle.
                            (1)
where   is the depth value of the ith row and jth column point of the 3D ROI,   is the corresponding mask value,  ,  ,   and   respectively denote the start row, end row, start column and end column. The parameters ,  ,   and   were set by experience. The reason we choose this region is that in the 3D ROI it appears to be relatively flat.
After getting the depth of the reference plane, we find the maximum depth,  , in a region denoted by the right rectangle in Fig. 3 (a) which starts at the 41st row and extends to the 160th row and from the 65th column to the 190th column. The MD can then be calculated easily by (2) as shown in Fig. 3 (b).
                                      (2)

\subsection{Horizontal Cross-section Area}
\label{ssec:methodology:hca}

To describe the shape of the 3D palmprint, we use a group of equidistant horizontal planes to cut the 3D ROI as shown in Fig. 4. Fig. 5 shows a 3D ROI and its contour cut by the equidistant horizontal planes. To render the shape clearly, Fig. 5 (a) only shows the 3D ROI image and hides the equidistant horizontal planes. In Fig. 5 (b), the blue curves denote deeper levels, the red curves denote higher levels and the remainder are medium levels. The HCA is defined as the area enclosed by the level curve. From Fig. 5 (b), we can see that most of the deeper level curves are enclosed and the areas are simply connected. These are more stable in response to noise or transformation.

Figure 4. Illustration of the 3D ROI crossed by horizontal planes.


(a)                                (b)
Figure 5. (a) A 3D ROI; (b) Its corresponding contour cutting by the equidistant horizontal planes.
To get a stable HCA, we take into consideration only the levels from the deepest point to the reference plane, defined in section 3.1. Suppose we divide this region into N levels. Every level   is described with a 200×200 matrix and calculated by (3).
          (3)
where   is the depth value of the ith row and jth column point of the 3D ROI and h is the palmprint depth defined by (2).
To make it more stable, we constrain every level growing from its previous level except the first level. That is
                      (4)
where “ ” denotes logical AND,   denotes a morphological dilation operation and   is a disk morphological structuring element whose size can be calculated by   (this is suitable for N = 8 by experience).
Fig. 6 shows an example of all the levels stacked together. Fig. 7 shows each of the levels separately.

Figure 6. An example of all the levels stacked together when N = 8.
After getting the cross-sectional levels  , the HCA,   can be easily calculated by
                                      (5)





Figure 7. The cross-sectional area feature (the top two rows are extracted from two samples collected from one palm; and the bottom two rows are extracted from two samples from another palm).

%!TEX root = featurecalc.tex
\subsection{Radial Line Length (RLL)}
\label{ssec:methodology:rll}

HCA is a coarse summary of the ROI characteristics. Different sample may have a similar area but dramatically different shape or contour of that area. To describe this shape characteristic, we need more elements to represent the ROI. The Radial Line Length (RLL) is then introduced for this purpose.

First, we calculate the centroid of the first level  $L^1$, thereafter we treat it as the reference point $P_{ref}$ for all levels. Then, from $P_{ref}$ we draw M radial lines which intersect with the contour of every level. RLL is defined as the distance from the intersection point to $P_{ref}$. The radial lines are distributed at equal angles. We record these radial lines from the inner layers to the outer layers starting with the horizontal direction by an $M$ by $N$ dimensional vector
\begin{equation}
R_i, i=1,2,\dots,M\times N
\end{equation}
where $M$ is the number of radial lines and $N$ is the number of cross-sections.

Figure ~\ref{fig:methodology:rll8} through ~\ref{fig:methodology:rll64} show some examples of RLL. This is a more detailed description than HCA as it takes the shape of each area into consideration.

\begin{figure}[htb]
\centering
\subfigure[$M=8$]{
    \includegraphics[width=0.4\textwidth]{ch-methodology/figures/rll8}
	\label{fig:methodology:rll8}
}\hspace{0.15\linewidth}
\subfigure[$M=16$]{
    \includegraphics[width=0.4\textwidth]{ch-methodology/figures/rll16}
	\label{fig:methodology:rll16}
}\\
\subfigure[$M=32$]{
    \includegraphics[width=0.4\textwidth]{ch-methodology/figures/rll32}
	\label{fig:methodology:rll32}
}\hspace{0.15\linewidth}
\subfigure[$M=64$]{
    \includegraphics[width=0.4\textwidth]{ch-methodology/figures/rll64}
	\label{fig:methodology:rll64}
}
\caption{$M=8,16,32,64$ radial lines starting from the reference point}
\label{fig:methodology:rllm}
\end{figure}


The above three features are mainly determined by the central region of the palm. This region is certainly contained by the ROI described in Section ~\ref{sec:methodology:roiextraction} which makes these features insensitive to translation and rotation. Although the RLL feature can be affected by rotation as the contours change smoothly, if the rotation is small then the variation of the RLL feature will also be small. Actually, there are some restricting pegs on the capture device which can guide the users to put their hands on the proper place as described in ~\cite{Zhang:2009dp}. Furthermore, we assume the user is cooperative when collecting data as we aim at civil rather than law enforcement applications.
