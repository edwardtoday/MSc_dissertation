\section{Region of Interest Extraction}
\label{sec:methodology:roiextraction}

Our definition and extraction procedure makes use of a 3D palmprint image containing 768×576 points captured using a structured-light imaging based 3D palmprint acquisition device [13]. First, we remove redundant and noisy boundary regions using a very simple Region of Interest (ROI) extraction process (Fig. 1). We segment a 400×400 points square that is respectively 68, 108, 234 and 134 points from the top, bottom, left and right boundaries of the 3D palmprint image as shown in Fig. 1 (a). Fig. 1 (b) shows the extracted ROI. After downsampling the 3D ROI to 200×200 points, we store it in a 200 by 200 matrix,  , where   is the depth value of the ith row and jth column point of the 3D ROI.

Our proposed 3D palmprint ROI extraction approach is much simpler than the one reported in [4] and the extracted shape features are not sensitive to translation and rotation, which is why we can use such a coarse ROI extraction. As the shape feature is a form of global feature, we extract as large an ROI as possible. Of course, such a large ROI may contain noisy data so we use a mask to remove the noisy data according to the gradient of the 3D data. If the gradient of the point, which is defined as  , is larger than a given threshold, the point is regarded as noisy data. Fig. 2 shows a 3D ROI which contains noisy data and its corresponding mask. We use a 200 by 200 matrix,  , to represent the mask, where   is noisy data and   is for other data.