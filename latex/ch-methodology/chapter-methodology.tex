
\chapter{Methodology\label{ch:methodology}}

The idea is to extract one or more features from the 3-D information as classifiers. Together with existing classifiers found in existing work in related fields [3,4,9], the new features will be combined to achieve a high performance classifier for personal authentication using Support Vector Machine (SVM).

The sample data has already been collected. There are 8,000 samples from 400 different palms with both hands. Samples of a palm are collected in two separate sessions with an average interval of one month.

![A 3-D palmprint sample](fig/sample.png "A 3-D palmprint sample")

![Contour view of the Region of Interest (ROI)](fig/roi_contour.png "Contour view of the Region of Interest (ROI)")

![Surface view of ROI](fig/roi.png "Surface view of ROI")

Experiments will be done using Matlab.

1. Extract features from each palmprint sample.
2. A subset of samples will be chosen as test set.
3. Train an authentication model based on the rest of samples.
4. Verify the samples in the test set with the trained model against their true identities.
5. Discuss the performance of the model.


