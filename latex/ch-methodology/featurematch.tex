\section{Feature Matching}
\label{sec:methodology:featurematch}

The classification of biometrics speeds up the identification process by reducing the number of comparisons that must be made. There are two kinds of classification techniques: exclusive classification and continuous classification. Both fingerprint [14] and palmprint classifications [15] make use of exclusive classification. The main problem of this technique is that it uses only a small number of classes and the samples are unevenly distributed between them, with more than 90\% of the samples being in just two or three classes. A further problem with exclusive classification is that when classification is performed automatically, it is necessary to handle errors and rejected samples gracefully, which is a hard problem in practice. In contrast, for continuous classification, samples are not partitioned into disjoint classes but rather associated with numerical vectors which represent features of the samples. These feature vectors are created through a similarity-preserving transformation so that similar samples are mapped into close points in the multi-dimensional space [19]. In this paper, we adopt the continuous classification technique. As the global features combining MD, HCA and RLL are high-dimensional, we reduce the dimensions using the LDA method. We then improve the efficiency of palmprint recognition by applying coarse-level matching and Ranking Support Vector Machine (RSVM) to the low dimensional vectors.

%!TEX root = featurematch.tex
\subsection{Dimension Reduction}
\label{ssec:methodology:lda}

%TODO find the citation
LDA is a state-of-the-art dimensionality reduction technique widely used in classification problems. The objective is to find the optimal projection which simultaneously minimizes the within-class distance and maximizes the between-class distance, thus achieving maximum discrimination (Here, the “class” is used to denote the identity of the subjects, e.g. the samples collected from one palm are regarded as one class). However, the traditional LDA requires the within-class scatter matrix to be nonsingular, which means the sample size should be large enough compared with its dimension, but is not always possible. In this paper, we therefore adopt the orthogonal LDA (OLDA) proposed in ~\cite{Ye:2005uu}, where the vectors of the optimal projection are calculated using the training database and the optimal projecting vectors are orthogonal to each other.

Suppose the 3D ROI has been divided to N levels and that M radial lines are used to represent the level contours. We can list the features as a column vector,
\begin{equation}
F=\{MD,A^1,A^2,\dots,A^N,R^1,R^2,\dots,R^{N\times M}\}
\end{equation}
with $1+N+N\times M$ rows. Given a training database which has $n$ samples and $k$ classes as $X=[X_1,X_2,\dots,X_k]$, where $X_i \in \mathbb{R}^{(1+N+N\times M)\times n}, i=1,2,\dots,k$ and $n=\sum \limits_{i=1}^k n_i$ , adopting OLDA ~\cite{Ye:2005uu} the optimal projection $W$ can be calculated as follows.

First, the within-class scatter matrix $S_w$, the between-class scatter matrix $S_b$ and total scatter matrix $S_t$ can be expressed as

\begin{equation}
S_w=H_w H_w^T, S_b=H_b H_b^T, S_t=H_t H_t^T
\end{equation}

where

\begin{equation}
H_w=\frac{1}{\sqrt{n}}\left[X_1 - m_1 \cdot e_1^T, \dots, X_k - m_k \cdot e_k^T \right]
\end{equation}

\begin{equation}
H_b=\frac{1}{\sqrt{n}}\left[\sqrt{n_1}(m_1-M),\dots,\sqrt{n_k}(m_k-m) \right]
\end{equation}

\begin{equation}
H_t=\frac{1}{\sqrt{n}}(X-m\cdot e^T)
\end{equation}

                                         (9)
where $m_i$ is the centroid of the $i$th class $X_i$, $m$ is the centroid of all the training samples $X$, $e_i=[1,1,\dots,1]^T \in \mathbb{R}^n, i=1,2,\dots,k$ and $e=[1,1,\dots,1]^T \in \mathbb{R}^n$.

After calculating $H_w$, $H_b$ and $H_t$, the reduced Singular Value Decomposition (SVD) is applied to $H_t$.

\begin{equation}
H_t \xrightarrow{\text{Reduced SVD}} U_r \Sigma_r V_r^T
\end{equation}

Denote $B=\Sigma_r^{-1} U_r^T H_b$ and compute the SVD of B.

\begin{equation}
B \xrightarrow{\text{SVD}} U_B \Sigma_B V_B^T
\end{equation}

Let

\begin{equation}
D=U_r\Sigma_r^{-1}U_B
\end{equation}

\begin{equation}
q=rank(B)
\end{equation}

and denote $D_q$ the first $q$ columns of the matrix D. Then, compute the QR decomposition of $D_q$.

\begin{equation}
D_q \xrightarrow{\text{QR decomposition}} QR
\end{equation}

where $Q$ is the desired orthogonal matrix and optimal projection, i.e. $W=Q$.

After getting the optimal projection $W$, we can map the $1+N+N\times M$ dimensional vector $F$ to a lower dimensional space

\begin{equation}
\tilde{F}=W^T F
\end{equation}
                                             (15)
where $\tilde{F}={f_1,f_2,\dots,f_{\Gamma}}$ is a $\Gamma$ dimensional vector with $\Gamma<1+N+N\times M$.

\subsection{Naive Matching}
\label{ssec:methodology:naive}

As the purpose of coarse-level matching is to speed up the identification during retrieval, we can regard it as a continuous classification approach. After mapping the global features’   dimensional vector  , the global features can be used to measure the similarity of two samples as follows:
                              (16)
In 3D palmprint identification, we can use the   dimensional global features to carry out coarse-level matching as shown in Fig. 9. If the testing sample passes coarse-level matching, it undergoes fine-level matching using 3D palmprint local features. If it does not pass, it moves on to the next sample in the database and so on until it has accessed the last sample in the database. From (16), we can see that coarse-level matching requires only   times of addition and multiplication which is much faster than fine-level matching using local features. Eq. (17) gives the fine-level matching by Mean Curvature Image (MCI) feature [13]:
                              (17)
where symbol “ ” represents the logical AND operation,   and   are the two binarized MCI features. To deal with the translation problem of ROI when calculating the matching score by (17), we will shift two, four, six and eight pixels of the test image along 8 directions: right, left, up, down, left-up, left-down, right-up and right-down, respectively. Adding the non-shift one, we will have   matching scores and the maximum one is selected. Suppose the size of the MCI feature is  , i.e.   and  , from (16) and (17) we can see that coarse-level matching is much faster than fine-level matching.

\subsection{Support Vector Machine}
\label{ssec:methodology:svm}

Coarse-level matching scheme is a simple and easy way to reduce retrieval times. It’s more useful for palmprint recognition if we can rank the candidate samples in the database in descending order according to the above global features. Searching for the closest matches to a given query vector in a large database is time consuming if the vector is even moderately high-dimensional. Various methods have been proposed to speed up the nearest neighbor retrieval, including hashing and tree structures [20]. However, the complexity of these methods grows exponentially with increasing dimensionality [21]. Therefore, we have adopted the Ranking Support Vector Machine (RSVM) method [18], inspired by the approaches of internet search engines, to rank the candidate samples in the database.

Figure 9. The flowchart of registration and recognition with coarse-level matching scheme.

Given a query   and a sample collection , the optimal retrieval system should return a ranking   that orders the samples in D according to their relevance to the query. In this paper, the query   and the sample   are the   dimensional global features as described above. In our approach, if a sample   is ranked higher than   in some ordering  , i.e.  , then  , otherwise  . Consider the class of linear ranking functions
                           (18)
where   is a weighted vector that is adjusted by learning and   is a pairwise distance function describing the match between   and   can be defined as  . Our goal is to find the optimal ranking function that will satisfy the maximum number of the following inequalities.
                                (19)
It is easier to solve this problem if it is converted into to the following SVM classification problem by introducing non-negative slack variable  .
Hence, the problem is to minimize:
                                   (20)
subject to:
                       (21)
                                              (22)
where C is a parameter that allows trading-off margin size against training error and   set by experience.
In the training stage, it is the inner-class samples of a test sample that should be ranked higher than the inter-class samples, e.g. inner-class samples rank is 1 and inter-class samples rank is 0. We input the ranks together with the   dimensional global features into the RSVM algorithm to learn the optimal ranking function  . Given a new query  , the samples in the database can be sorted by their value of
                                  (23)
