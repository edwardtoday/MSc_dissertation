%!TEX root = chapter-conclusion.tex
\section{Summary}
% In this work, we explain how to use the puthesis.cls class file and the accompanying template.

In this dissertation, the authentication process are improved using 3D features. The features are stable in samples from a single person over time and distinguishable among samples from different people.

Three features adopted are Maximum Depth of palm center, Horizontal Cross-section Area and Radial Line Length from the centroid to the boundary of 3D palmprint horizontal cross-section of different levels. These cannot be extracted from 2D palmprints and are not correlated with local features, such as line and texture features. To make these features efficient for use in coarse classification, we treat them as a multi-dimensional vector and use OLDA to map it to a lower dimensional space.

We then improve the efficiency of 3D palmprint recognition using two proposed approaches, coarse-level matching and RSVM, both of which significantly reduce the penetration rate during retrieval.

Experiments are conducted on an existing 3D palmprint database of 8,000 samples. The results show that the proposed method is able to achieve an reasonable performance.
