%!TEX root = ../thesis.tex
\chapter{Core Matlab Code\label{ch:code}}



\singlespacing

Create large arrays without asking for contiguous memory.

\begin{lstlisting} %\begin{minted}[fontsize=\small,fontfamily=courier,frame=lines,linenos]{matlab}
function result = createArrays(nArrays, arraySize, datatype)
%CREATEARRAYS
% This creates an cell array which does NOT require
% contiguous memory.
%
% To use it:
%   myArray = createArrays(numberOfArrays, [x y], 'single');
%
% To access the elements:
%   myArray{1}{2,3} = 10;
%   myArray{1} = zeros(500, 800, 'single');
    result = cell(1, nArrays);
    for i = 1 : nArrays
        result{i} = zeros(arraySize, datatype);
    end
end
\end{lstlisting} %\end{minted}
\clearpage

Code to find the reference plane.

\begin{lstlisting} %\begin{minted}[fontsize=\small,fontfamily=courier,frame=lines,linenos]{matlab}
function [ d_ref ] = find_ref_plane( roi )
%FIND_REF_PLANE Find the depth of reference plane of an ROI
%   Detailed explanation goes here

    [~,roi_size] = size(roi);

    refplane = roi(1:(roi_size*0.25),
     (roi_size*0.25):(roi_size*0.9));

    d_ref = max(max(refplane));

end
\end{lstlisting} %\end{minted}
\clearpage

Code to extract the X coordinates from original data.
\begin{lstlisting} %\begin{minted}[fontsize=\small,fontfamily=courier,frame=lines,linenos]{matlab}
clear;

prefix = '..\..\3D_palm';

%% File to process
load_dir = [prefix filesep '3DPalm_xyz'];
file_mask = [load_dir filesep '*.dat'];
file_list = dir(file_mask);
file_list = char({file_list.name});

%% Get X coordinates from file
tic;
dat_file = reshape(dlmread([load_dir filesep file_list(1,:)]),
               768,576,3);
palmX=dat_file(:,1,1);

%% Save result to file
save_dir = ['output'];
save([save_dir filesep '3Dpalm_x.mat'],'palmX');
toc;
\end{lstlisting} %\end{minted}
\clearpage

Code to remove outliers from a vector.

\begin{lstlisting} %\begin{minted}[fontsize=\small,fontfamily=courier,frame=lines,linenos]{matlab}
function [b,idx,outliers] = deleteoutliers(a,alpha,rep);
% [B, IDX, OUTLIERS] = DELETEOUTLIERS(A, ALPHA, REP)
%
%   For input vector A, returns a vector B with
%   outliers (at the significance level alpha) removed.
%   Also, optional output argument idx returns the
%   indices in A of outlier values. Optional output
%   argument outliers returns the outlying values in A.
%
%   ALPHA is the significance level for determination of outliers.
%   If not provided, alpha defaults to 0.05.
%
%   REP is an optional argument that forces the replacement of removed elements with NaNs to
%   presereve the length of a.
%
%   This is an iterative implementation of the Grubbs
%   Test that tests one value at a time. In any given
%   iteration, the tested value is either thehighest
%   value, or the lowest, and is the value that is
%   furthest from the sample mean. Infinite elements
%   are discarded if rep is 0, or replaced with NaNs
%   if rep is 1.

if nargin == 1
    alpha = 0.05;
    rep = 0;
elseif nargin == 2
    rep = 0;
elseif nargin == 3
    if ~ismember(rep,[0 1])
        error('Please enter a 1 or a 0 for optional
               argument rep.')
    end
elseif nargin > 3
    error('Requires 1,2, or 3 input arguments.');
end

if isempty(alpha)
    alpha = 0.05;
end

b = a;
b(isinf(a)) = NaN;

%Delete outliers:
outlier = 1;
while outlier
    tmp = b(~isnan(b));
    meanval = mean(tmp);
    maxval = tmp(find(abs(tmp-mean(tmp))==
                      max(abs(tmp-mean(tmp)))
                     ));
    maxval = maxval(1);
    sdval = std(tmp);
    tn = abs((maxval-meanval)/sdval);
    critval = zcritical(alpha,length(tmp));
    outlier = tn > critval;
    if outlier
        tmp = find(a == maxval);
        b(tmp) = NaN;
    end
end

if nargout >= 2
    idx = find(isnan(b));
end
if nargout > 2
    outliers = a(idx);
end
if ~rep
    b=b(~isnan(b));
end
return

function zcrit = zcritical(alpha,n)
%ZCRIT = ZCRITICAL(ALPHA,N)
% Computes the critical z value for rejecting outliers
% (GRUBBS TEST)

tcrit = tinv(alpha/(2*n),n-2);
zcrit = (n-1)/sqrt(n)*(sqrt(tcrit^2/(n-2+tcrit^2)));
\end{lstlisting} %\end{minted}
\clearpage

Code for batch feature extraction.

\begin{lstlisting} %\begin{minted}[fontsize=\small,fontfamily=courier,frame=lines,linenos]{matlab}
clear;

%% Files to process
save_dir = ['output'];

use_original_data = true;
% Toggle this for data source selection
% use_original_data = false;

if (use_original_data)
    data_dir = ['..' filesep 'rawdata'];
    file_mask = [data_dir filesep '*.zonly'];
    sample_width = 768;
    sample_height = 576;
    roi_size = 400;
else
    data_dir = ['..' filesep 'rawdata' filesep 'roi'];
    file_mask = [data_dir filesep '*.dat'];
    sample_width = 128;
    sample_height = 128;
    roi_size = 128;
end

file_list = dir(file_mask);
file_list = char({file_list.name});

num_of_samples = length(file_list);
sample_per_person = 10;
num_of_people = num_of_samples/sample_per_person;
% Toggle this for limited number of files
// num_of_people = 100;

feature_dimension = roi_size;
% feature_dimension = roi_size*roi_size;

features = zeros(num_of_people*sample_per_person,
                 feature_dimension);

%% Parallel read files
tic;
for current_person = 1:num_of_people
    disp(['Loading data from ' num2str(current_person) \
        ' of ' num2str(num_of_people) ' people.']);
    for current_sample = 1:sample_per_person
        file_id = (current_person - 1) * 10 +
                   current_sample;
        sample_filename = strtrim([data_dir filesep
                            file_list(file_id,:)]);

        mat = file2matrix(sample_filename,
                          sample_width,
                          sample_height);

        if (use_original_data)
            % Crop ROI from original data
            mat = mat(235:(234+roi_size),69:(68+roi_size));
        end;

        % Extract feature for this sample
        features(file_id,:) = calc_feature(mat);
    end
end

save(['output' filesep 'features.mat'], 'features');
toc;
\end{lstlisting} %\end{minted}
\clearpage

Code for curvature extraction.

\begin{lstlisting} %\begin{minted}[fontsize=\small,fontfamily=courier,frame=lines,linenos]{matlab}
function [K,H,Pmax,Pmin] = surfature(X,Y,Z),
% SURFATURE -  COMPUTE GAUSSIAN AND MEAN CURVATURES
%   OF A SURFACE
%
%   [K,H] = SURFATURE(X,Y,Z),
%           WHERE X,Y,Z ARE 2D ARRAYS OF POINTS
%   ON THE SURFACE.  K AND H ARE THE GAUSSIAN AND MEAN
%   CURVATURES, RESPECTIVELY.
%
%   SURFATURE RETURNS 2 ADDITIONAL ARGUEMENTS,
%
%   [K,H,Pmax,Pmin] = SURFATURE(...), WHERE Pmax AND
%       Pmin ARE THE MINIMUM AND MAXIMUM CURVATURES AT
%       EACH POINT, RESPECTIVELY.


% First Derivatives
[Xu,Xv] = gradient(X);
[Yu,Yv] = gradient(Y);
[Zu,Zv] = gradient(Z);

% Second Derivatives
[Xuu,Xuv] = gradient(Xu);
[Yuu,Yuv] = gradient(Yu);
[Zuu,Zuv] = gradient(Zu);

[Xuv,Xvv] = gradient(Xv);
[Yuv,Yvv] = gradient(Yv);
[Zuv,Zvv] = gradient(Zv);

% Reshape 2D Arrays into Vectors
Xu = Xu(:);   Yu = Yu(:);   Zu = Zu(:);
Xv = Xv(:);   Yv = Yv(:);   Zv = Zv(:);
Xuu = Xuu(:); Yuu = Yuu(:); Zuu = Zuu(:);
Xuv = Xuv(:); Yuv = Yuv(:); Zuv = Zuv(:);
Xvv = Xvv(:); Yvv = Yvv(:); Zvv = Zvv(:);

Xu          =   [Xu Yu Zu];
Xv          =   [Xv Yv Zv];
Xuu         =   [Xuu Yuu Zuu];
Xuv         =   [Xuv Yuv Zuv];
Xvv         =   [Xvv Yvv Zvv];

% First fundamental Coeffecients of the surface (E,F,G)
E           =   dot(Xu,Xu,2);
F           =   dot(Xu,Xv,2);
G           =   dot(Xv,Xv,2);

m           =   cross(Xu,Xv,2);
p           =   sqrt(dot(m,m,2));
n           =   m./[p p p];

% Second fundamental Coeffecients of the surface (L,M,N)
L           =   dot(Xuu,n,2);
M           =   dot(Xuv,n,2);
N           =   dot(Xvv,n,2);

[s,t] = size(Z);

% Gaussian Curvature
K = (L.*N - M.^2)./(E.*G - F.^2);
K = reshape(K,s,t);

% Mean Curvature
H = (E.*N + G.*L - 2.*F.*M)./(2*(E.*G - F.^2));
H = reshape(H,s,t);

% Principal Curvatures
Pmax = H + sqrt(H.^2 - K);
Pmin = H - sqrt(H.^2 - K);
\end{lstlisting} %\end{minted}
\clearpage

Code to test feature performance.

\begin{lstlisting} %\begin{minted}[fontsize=\small,fontfamily=courier,frame=lines,linenos]{matlab}
%% Load features
load(['output' filesep 'features.mat']);

[total,~] = size(features);

%% Split into training set and testing set
training_id=unique([
    1:10:total
    2:10:total
    3:10:total
    4:10:total
    5:10:total
    6:10:total
    ]);

trainingset = features(training_id,:);

test_id=unique([
    7:10:total
    8:10:total
    9:10:total
    10:10:total
    ]);

testingset = features(test_id,:);

%% Find match for each test input
nearest_id = knnsearch(trainingset, testingset, 'k', 10);

%% Interpret the match to person
[total_training,~] = size(trainingset);
training_sample_per_person = total_training /
                             total * 10;
nearest_person = ceil(nearest_id/
                 training_sample_per_person);

%% Check performance
[total_testing,~] = size(testingset);
testing_sample_per_person = total_testing / total * 10;

for i = 1:total_testing
    correct(i) = eq(nearest_person(i),
               ceil(i/testing_sample_per_person));
end

correct = correct';

accuracy = sum(correct)/total_testing
\end{lstlisting} %\end{minted}
\clearpage

Code to create mask according to gradient.

\begin{lstlisting} %\begin{minted}[fontsize=\small,fontfamily=courier,frame=lines,linenos]{matlab}
sizeH = 200; %
sizeW = 200; %
lnum = 8;

pixDis = 0.28;
% load facesMatrix64.mat;
temp = pixDis * ((sizeH-2)/2+0.5);
[X,Y] = meshgrid(-temp : pixDis : temp);

fid = fopen('../rawdata/Sub3D_I_3_0.dat', 'r');
% fid = fopen('../rawdata/Sub3D_II_100_0.dat', 'r');

Z = fread(fid, [sizeH,sizeW], 'double');
fclose(fid);

[fx, fy] = gradient(Z);
fxy = fx.^2 + fy.^2;

noiseP = find(fxy>0.1);
% 0.1 used for corrected and smoothed Sub3D,
% 1 used for original Sub3D

%1 for valid point, 0 for invalid point
flag = ones(sizeH, sizeW);

flag(noiseP) = 0;
flag = 1 - flag;
se = strel('disk',5);
flag = imdilate(flag, se);
flag = 1 - flag;
noiseP = find(flag==0);
meanZ = sum(sum(Z .* flag)) /
        (sizeH*sizeW - length(noiseP));
Z(noiseP) = 5;
%neend't smooth
% Z = smooth(Z, 7);

%%for show the mask which get rid of bad quality region
% flag(1,:) = 0; flag(end,:) = 0;
% flag(:,1) = 0; flag(:,end) = 0;
% imshow(flag');
\end{lstlisting} %\end{minted}
\clearpage

Code to determine the reference plane and MD.

\begin{lstlisting} %\begin{minted}[fontsize=\small,fontfamily=courier,frame=lines,linenos]{matlab}
%calculate the reference 0 plane
% use Z(6:35, 65:136) for calculate the mean value as reference 0
refRect = Z(6:35, 65:136);
refFlag = flag(6:35, 65:136);
refVal = sum(sum(refRect .* refFlag)) / sum(sum(refFlag));
Z = Z - refVal;

%search the min point in Z(65:190, 41:160)
rectforMin = Z(65:190, 41:160);
rectforMin(1:35, 1:25) = 5;
rectforMin(1:35, end-25:end) = 5;
% flagforMin = Z(65:190, 41:160);

palmH = min(min(rectforMin));
ind = find(Z == palmH);
% Z(ind) = 5;
\end{lstlisting} %\end{minted}
\clearpage

Code to grow each level from the reference point.

\begin{lstlisting} %\begin{minted}[fontsize=\small,fontfamily=courier,frame=lines,linenos]{matlab}
%find the level regions from 0 to deepest point
%find the region > 0
palmH = -palmH;
Z = -Z;
Ln = 8;
step = palmH / Ln;
levelH = [0:step:palmH];

for i = 1:Ln
    L0 = zeros(sizeH, sizeW);
    L0( find( Z>=levelH(Ln-i+1) ) ) = 1;
    %the 1st level
    if i==1
        [L,num] = bwlabel(L0);

        if num > 1
            for j = 1:num
                indL = find(L==j);
                if length(find(indL==ind)) > 0
                    L0(:,:) = 0;
                    L0(indL) = 1;
                end
            end
        end
        Lp = L0;
        L0 = logical(L0');
        [x, y] = find(L0);
        temp = find(L0);
        mx = mean(x);
        my = mean(y);

%         saveIm(L0, i);
        figure;
        imshow(L0);
    else
        %dilate the Lp (previous level) and than & with L0
        se = strel('disk',35 - 3*i);
        L1 = imdilate(Lp, se);
        L0 = L0 & L1;
        Lp = L0;
        L0 = L0';
%         saveIm(L0, i);
        figure;
        imshow(L0);
    end
end
\end{lstlisting} %\end{minted}
\clearpage


Code to show the growed levels.

\begin{lstlisting} %\begin{minted}[fontsize=\small,fontfamily=courier,frame=lines,linenos]{matlab}
%for show
im_level = zeros(sizeH, sizeW); %for show the levels
%find the level regions from 0 to deepest point
%find the region > 0
palmH = -palmH;
Z = -Z';
Ln = 8;
step = palmH / Ln;
levelH = [0:step:palmH];
for i = 1:Ln
    L0 = zeros(sizeH, sizeW);
    L0( find( Z>=levelH(Ln-i+1) ) ) = 1;

    %the 1st level
    if i==1
        [L,num] = bwlabel(L0);
        if num > 1
            for j = 1:num
                indL = find(L==j);
                if length(find(indL==ind)) > 0
                    L0(:,:) = 0;
                    L0(indL) = 1;
                end
            end
        end
        Lp = L0;
        [x, y] = find(L0);
        temp = find(L0);
        mx = round(mean(x));
        my = round(mean(y));
        im_level(temp) = 155;

%         saveIm(L0, i);
%         figure;
%         imshow(L0);
    else
        % dilate the Lp (previous level) and then
        % bitwise & with L0
        se = strel('disk',35 - 3*i);
        L1 = imdilate(Lp, se);
%         figure; imshow(Lp);
%         figure; imshow(L1);
        L0 = L0 & L1;
%         figure; imshow(L0);
        Ltemp = L0 - Lp;
%         figure; imshow(Ltemp);
        Lp = L0;
        temp = find(Ltemp);
        im_level(temp) = 255-(i-1)*30;
%         saveIm(L0, i);
%         figure;
%         imshow(L0);
    end
end

im_level = uint8(im_level);
% imshow(im_level);
% imwrite(im_level, 'figures/im_levels.bmp');
\end{lstlisting} %\end{minted}
\clearpage

Other functions.

\begin{lstlisting} %\begin{minted}[fontsize=\small,fontfamily=courier,frame=lines,linenos]{matlab}
function saveIm(data, nlevel)
filename = ['../rawdata/bmp/Sub3D_I_4_9',
            '_' num2str(nlevel), '.bmp'];
imwrite(data, filename);


function [z] = smooth(z, fs)
% z -- 128*128
% n -- filter size
t = floor(fs/2)-1;
for i = 0:t;
    [m,n] = size(z);
    z = [z(:, 2*i+1) z(:,:) z(:, n-2*i)];
end

for i = 0:t;
    [m,n] = size(z);
    z = [z(2*i+1, :); z(:,:); z(m-2*i, :)];
end

h = ones(fs,fs)/(fs*fs);
z = filter2(h, z, 'valid');
\end{lstlisting} %\end{minted}
\clearpage
