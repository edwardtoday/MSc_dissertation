%!TEX root = thesis.tex

Three dimensional palmprint has proved to be a significant biometrics for personal authentication. Three dimensional palmprints are harder to counterfeit than two dimensional palmprints and more robust to variations in illumination and serious scrabbling on the palm surface. Concentration of most previous work on recognition with three dimensional palmprint data has been on local features such as texture and lines.

In this dissertation, the authentication process are improved using three dimensional features. The features are stable in samples from a single person over time and distinguishable among samples from different people. Three features adopted are Maximum Depth of palm center, Horizontal Cross-section Area and Radial Line Length from the centroid to the boundary of three dimensional palmprint horizontal cross-section of different levels. The feature set is combined as a column feature vector for matching. Support Vector Machine is introduced for higher efficiency.

Experiments are conducted on an existing three dimensional palmprint database of 8,000 samples. The results show that the proposed method is able to achieve an reasonable performance.


% \textbf{Keywords}: 3D palmprint identification, 3D palmprint verification, SVM
