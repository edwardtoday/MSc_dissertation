%!TEX root = thesis.tex

Three Dimensional (3D) palmprint has proved to be a significant biometrics for personal authentication. 3D palmprints are harder to counterfeit than 2D palmprints and more robust to variations in illumination and serious scrabbling on the palm surface. Previous work on 3D palmprint recognition has concentrated on local features such as texture and lines.

% a blank line as required
\vspace{12pt}

In this dissertation, the authentication process are improved using 3D features. The features are stable in samples from a single person over time and distinguishable among samples from different people. Three features adopted are Maximum Depth of palm center, Horizontal Cross-section Area and Radial Line Length from the centroid to the boundary of 3D palmprint horizontal cross-section of different levels. The feature set is combined as a column feature vector for matching. Support Vector Machine is introduced for higher efficiency.

% a blank line as required
\vspace{12pt}

Experiments are conducted on an existing 3D palmprint database of 8,000 samples. The results show that the proposed method is able to achieve an reasonable performance.


% \textbf{Keywords}: 3D palmprint identification, 3D palmprint verification, SVM
