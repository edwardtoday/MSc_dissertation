%!TEX root = ../thesis.tex
\chapter{Introduction\label{ch:intro}}

Palmprint has been increasingly recognized as unique and stable biometric characteristics for personal authentication. In the past decade, various methods based on two dimensional (2-D) palmprint have been studied in depth. The 2-D recognition techniques have proved to achieve high accuracy~\cite{Kong:2009hj}.

In recent years, three dimensional (3-D) palmprint recognition devices emerge and are quite promising because of the additional depth information gathered.

Although the devices has been out for more than two years, most previous matching algorithms treat 3-D information as a supplement to 2-D texture images and used joint matching techniques to increase accuracy~\cite{Li:2011ur, Li:2010en, Zhang:2009dp, Zhang:2008kc, Zhang:2010uu}. Authentication with only the 3-D information has not been thoroughly studied. The amount of useful information carried by the 3-D data is still under investigation.

There are two major challenges:

\begin{enumerate}
\item 3-D devices, compared to 2-D ones, are lower in resolution.
\item The depth values are susceptible to movements of human hands and are therefore less stable than 2-D texture information of palmprints.
\end{enumerate}

David et al. explore a 3-D palmprint recognition approach by exploiting the 3-D structural information of the palm surface~\cite{Zhang:2009dp, Zhang:2008kc}. The structured light imaging is used to acquire the 3-D palmprint data, from which several types of unique features, including mean curvature image, Gaussian curvature image, and surface type, are extracted. A fast feature matching and score-level fusion strategy is proposed for palmprint matching and classification. Wei et al. propose an efficient joint 2D and 3D palmprint matching scheme~\cite{Li:2010en}. The principal line features and palm shape features are extracted and used to accurately align the palmprint, and a couple of matching rules are defined to efficiently use the 2D and 3D features for recognition. The experiments show that the proposed scheme can greatly improve the performance of palmprint verification. Wei et al. also present an efficient scheme for 3-D palmprint recognition~\cite{Li:2011ur}. They extract both line and orientation features after calculating and enhancing the mean-curvature image of the 3-D palmprint data. The two types of features are then fused at either score level or feature level for the final 3-D palmprint recognition.

Existing work has been done to utilize the 3-D information for palmprint classification and sorting. The global features proposed for that purpose are fast in matching speed but low in accuracy compared to 2-D techniques.

Three features are extracted from 3D palmprint depth data: Maximum Depth, Horizontal Cross-section Area and Radial Line Length. Theses features are combined together as a brief description of a 3D palmprint sample. Viewing the feature set as a column vector, various data processing techniques can be applied to improve the recognition efficiency such as dimension reduction and machine learning.

The rest of this work is organized as follows. Chapter \ref{ch:pastwork} briefly introduces the existing research efforts related to this topic. Chapter \ref{ch:methodology} describes the entire data processing procedures to extract features and use them for 3D palmprint recognition. Chapter \ref{ch:experiment} shows some experiment results. Chapter \ref{ch:conclusion} concludes the work and points out potentials for some future work.
